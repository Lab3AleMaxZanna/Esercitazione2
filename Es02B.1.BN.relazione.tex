\documentclass[10pt,a4paper]{article}
\usepackage[utf8]{inputenc}
\usepackage[italian]{babel}
\usepackage{amsmath}
\usepackage{amsfonts}
\usepackage{amssymb}
\usepackage{graphicx}
\usepackage[left=2cm,right=2cm,top=2cm,bottom=2cm]{geometry}
\newcommand{\rem}[1]{[\emph{#1}]}
\newcommand{\exn}{\phantom{xxx}}

\author{Gruppo 1.BN \\ Massimo Bilancioni, Alessandro Foligno }
\title{Es02B:Circuito RC - Filtri passivi}
\begin{document}
\date{4 ottobre 2018}
\maketitle

\setcounter{section}{1}
\section{Filtro passa-basso}
I valori misurati sono: $R_1  = (3.24\pm 0.03 )K\Omega$ e  $C_1  = (9.66\pm 0.40 )nF$.

a) La frequenza di taglio teorica è  $f_T = 1/2\pi R_1 C_1$ che in base ai valori viene $f_T = 5.09\pm 0.21)KHz$


\paragraph{2.b Partitore con resistenze da circa 1~k}
Valori misurati $R_1$ e $R_2$ e valore atteso di $A_\mathrm{exp}$:
\[
R_1 = (0.988  \pm0.008  ) \,\mathrm{k}\Omega, \quad
R_2 = (1.187 \pm 0.01 ) \,\mathrm{k}\Omega, \quad
A_\mathrm{exp} = ( 0.544 \pm 0.002 ) 
\]


\begin{table}[h]
\centering
\begin{tabular}{|c|c|c|c|c|c|}
\hline 
VIN& $\sigma$ VIN  &VOUT	 & $\sigma$ VOUT& VOUT/VIN & $\sigma$ VOUT/VIN \\
\hline 
1.121 & 0.005 & 0.612 & 0.003 & 0.546 & 0.005\\
1.944& 0.010 & 1.062& 0.005 & 0.546 & 0.005\\

3.59& 0.02 & 1.955 & 0.010 & 0.544 & 0.005\\

4.68 & 0.03 & 2.55 & 0.02 & 0.545 & 0.005\\
6.46 & 0.03 & 3.52 & 0.02 & 0.545& 0.005\\
6.51 & 0.03& 3.55 & 0.02& 0.545 & 0.005 \\
8.49 & 0.04 & 4.63& 0.03 & 0.545& 0.005\\
9.89 & 0.05 & 5.40 & 0.03 & 0.546 & 0.005\\


\hline 
\end{tabular} 
\caption{(2.b) Partitore di tensione con resistenze da circa 1k. Tutte le tensioni in V.\label{t:par1}}
\end{table}



Il valore calcolato $A_{fit}$ dal fit lineare è $ 0.5452\pm0.0003$, dal grafico (e da un  calcolo del $\chi^2$) si vede come gli errori sono piuttosto sovrastimati. Questo  è dovuto al fatto che la maggior parte dell'errore del multimetro sui Voltaggi è dovuto a un fattore di scala, che risulta ininfluente nel rapporto fra due misure effettuate con lo stesso fondoscala  (che è il caso della quasi totalità delle misure).


\paragraph{2.c Partitore con resistenze da circa 4M}
Valori misurati $R_1$ e $R_2$ e valore atteso di $A_\mathrm{exp}$:
\[
R_1 = ( 4.66 \pm 0.05 ) \,\mathrm{M}\Omega, \quad
R_2 = ( 3.25 \pm 0.04 ) \,\mathrm{M}\Omega, \quad
A_\mathrm{exp} = ( 0.41 \pm0.02 ) 
\]


\begin{table}[h]
\centering
\begin{tabular}{|c|c|c|c|c|c|}
\hline 
VIN& $\sigma$ VIN  &VOUT	 & $\sigma$ VOUT& VOUT/VIN & $\sigma$ VOUT/VIN \\
\hline 
1.965 & 0.010 & 0.675 & 0.004 & 0.345 & 0.003 \\
2.58 & 0.02 & 0.891 & 0.004 & 0.345 & 0.003 \\
5.12 & 0.03 & 1.763 & 0.008 & 0.344 & 0.003 \\
3.97 & 0.02 & 1.367 & 0.005 & 0.344 & 0.003 \\
6.93 & 0.04 & 2.39 & 0.02 & 0.345 & 0.003 \\
8.67 & 0.04 & 2.98 & 0.02 & 0.344 & 0.003 \\
10.12 & 0.05 & 3.49 & 0.02 & 0.345 & 0.003\\


\hline 
\end{tabular} 
\caption{(2.c) Partitore di tensione con resistenze da circa 4M. Tutte le tensioni in V.\label{t:par2}}
\end{table}




Come prima, dal grafico stimiamo $A_{fit}=0.3443\pm0.0004$, non compatibile col valore atteso; ciò significa che il modello è sbagliato e la resistenza del tester non è trascurabile



\paragraph{2.d Resistenza di ingresso del tester}
Usando il modello mostrato nella scheda si ottiene
\[ \frac{R_1}{R_T} =  \frac{V_{IN}}{V_{OUT}} - (1 +  \frac{R_1}{R_2} )
\]

Con i dati del punto 2.b si ottiene
\[ R_1/R_T = 0.003 \pm  0.007   \rightarrow  R_T > 100 k\Omega
\]


Con i dati del punto 2.c si ottiene
\[ R_1/R_T = 0.47  \pm0.02  \exn   \rightarrow  R_T = (9.9 \pm  0.5)  M\Omega
\]


L'incertezza è sovrastimata perchè gli errori di $R_1$ e $R_2$ non sono scorrelati, essendo misure fatte con lo stesso fondoscala. In particolare considerando l'errore sistematico come un fattore di scala viene che l'incertezza sul rapporto dipende solo dall'incertezza sull'ultimo digit.

\section{Uso dell'oscilloscopio}

\paragraph{3.b Misure di tensione} 
Vengono ripetute le misure del punto 2.c  ma con pochi punti e senza grafico
\begin{table}[h]
\centering
\begin{tabular}{|c|c|c|c|c|c|}
\hline 
VIN& $\sigma$ VIN  &VOUT	 & $\sigma$ VOUT& VOUT/VIN & $\sigma$ VOUT/VIN \\
\hline 

3.0 & 0.02 & 0.527 & 0.016 & 0.176 & 0.010 \\
4.26 & 0.02 & 0.635 & 0.02 & 0.149 & 0.008 \\
5.58 & 0.03 & 0.866 & 0.025 & 0.155 & 0.008 \\
6.86 & 0.03 & 1.01 & 0.03 & 0.147 & 0.008\\
7.67 & 0.04 & 1.17 & 0.04 & 0.153 & 0.009\\
8.31 & 0.04 & 1.24 & 0.04 & 0.149 & 0.008\\
9.91 & 0.05 & 1.45 & 0.07 & 0.146 & 0.008 \\
\hline 
\end{tabular} 
\caption{(3.b) Partitore di tensione con resistenze da circa 4M, misura con oscilloscopio. Tutte le tensioni in V.}
\end{table}

La stima di $V_{out}/{V_{in}}$ dà come risultato $0.153\pm0.006$



\paragraph{3.d Impedenza di ingresso dell'oscilloscopio} Si ripete l'analisi del punto 2.d

\[ R_1/R_{IN} = 4.10  \pm  0.26  \rightarrow  R_{IN} = (1.14 \pm  0.06)  M\Omega
\]


\section{Misure di frequenza e tempo}

\paragraph{4.b Misure di frequenza}
Misure con onda sinusoidale
\begin{table}[h]
\centering
\begin{tabular}{|c|c|c|c|c|c|}
\hline 
Periodo T ($\mu$ s)& $\sigma$ T ($\mu$ s)  &Frequenza f (KHz) & $\sigma$ f (KHz) & Misura oscilloscopio (KHz) & Differenza (KHz)\\
\hline 
$1.56\cdot 10^3$& 10 & 0.641 & 0.004 & 0.640 &0.001\\
157 & 1 & 6.36 & 0.04& 6.33 &0.03\\
15.5& 0.1 & 64.5 & 0.4 & 64.1 &0.4 \\
1.15 & 0.01 & 869& 5 & 870 &1 \\
\hline 
\end{tabular} 
\caption{(4.b) Misura di frequenza di onde sinusoidali  e confronto con misurazione interna dell'oscilloscopio }
\end{table}


\section{Trigger dell'oscilloscopio}
\paragraph{5.b Segnale pulse}
Misure con segnale pulse del generatore di onde


\clearpage
\section{Conclusioni e commenti finali}
Il valore $R_T = (9.9 \pm  0.5)  M\Omega$ è in accordo con il valore di $10M\Omega$  riportato sul manuale del tester.

La deviazione di $R_{IN}  = (1.14 \pm  0.06)  M\Omega$ dal valore nominale dell'impedenza d'ingresso dell'oscilloscopio $R_{nom}= 1M\Omega \pm 2\% $ può essere dovuta al numero insufficiente di misure prese.

\section*{Dichiarazione}
I firmatari di questa relazione dichiarano che il contenuto della relazione \`e originale, con misure effettuate dai membri del gruppo, e che tutti i firmatari hanno contribuito alla elaborazione della relazione stessa.

\end{document}