\documentclass[10pt,a4paper]{article}
\usepackage[utf8]{inputenc}
\usepackage[italian]{babel}
\usepackage{amsmath}
\usepackage{amsfonts}
\usepackage{amssymb}
\usepackage{graphicx}
\usepackage[left=2cm,right=2cm,top=2cm,bottom=2cm]{geometry}
\newcommand{\rem}[1]{[\emph{#1}]}
\newcommand{\exn}{\phantom{xxx}}

\author{Gruppo 1.BN \\ Massimo Bilancioni, Alessandro Foligno }
\title{Es02B:Circuito RC - Filtri passivi}
\begin{document}
\date{4 ottobre 2018}
\maketitle


\section{Filtro Passa-basso}
\subsection{}
\label{passa-basso}

I valori misurati sono: $R_1  = (3.24\pm 0.03 )K\Omega$ e  $C_1  = (9.7\pm 0.4 )nF$.

a) La frequenza di taglio teorica è  $f_T = 1/2\pi R_1 C_1$ che in base ai valori viene $f_T = 5.09\pm 0.21)KHz$\\
b) A bassa frequenza la funzione di guadagno vale \[1-\frac{f^2}{2ft^2}\] dove la condizione è \[2\pi RCf\ll1\]
c)A $2kHz$ il guadagno vale \[A=0.93\pm0.01\]
d)A $20kHz$ il guadagno vale \[A=0.24\pm0.01\]. Notiamo che già non siamo più nel regime di basse frequenze per usare la formula del punto b).
\subsection{}{2.b Partitore con resistenze da circa 1~k}
Valori misurati $R_1$ e $R_2$ e valore atteso di $A_\mathrm{exp}$:
\[
R_1 = (0.988  \pm0.008  ) \,\mathrm{k}\Omega, \quad
R_2 = (1.187 \pm 0.01 ) \,\mathrm{k}\Omega, \quad
A_\mathrm{exp} = ( 0.544 \pm 0.002 ) 
\]




\subsection{}

La misura del tempo di salita è $ t_{sal} = (70\pm 5)\mu s$
\[ f_t = \frac{1.1}{\pi t_{sal}} = (5.00 \pm 0.36) kHz \]

\subsection{}
a) l'impedenza di ingresso del circuito è
\begin{itemize}

\item a bassa frequenza  infinita, è un circuito aperto per la presenza del condensatore.
\item ad alta frequenza $Z_{circuito}\sim R_1$, perchè l'impedenza del condensatore è trascurabile
\item alla frequenza di taglio  $Z_{circuito} = R_1(1-j)$.

\end{itemize}

b)  Se $R_c$ è la resistenza di carico e $A_1$ la funzione di trasferimento del passa-basso senza il carico,  la nuova funzione di trasferimento diventa:
\[ A_{1c} = v_{out}/v_{in}= \frac{A_1}{1+\frac{R_1}{R_c}A_1}\]

Si vede che $A_{1c} \sim A_{1}$ nel limite in cui $R_1 \ll R_c$, che risulta ragionevolmente vero per $R_c = 100 k\Omega$.

Nel caso in cui $R_c = 10 \, k\Omega$, $A_{1c}$ è sensibilmente diversa: 

max $|A_{1c}|= \frac{1}{1+R_1/R_C}=0.755$ (il guadagno massimo è minore di 1) e la frequenza di taglio  aumenta $f_{tc}= 1.18 f_{t}$.

\section{Filtro Passa-banda}
\subsection{}
Ivalori di $R_1$ e $C_1$ sono quelli della sezione \ref{passa-basso}, mentre i valori misurati di $C_2$ e $R_2$ sono $R_2  = (3.28 \pm 0.03 )K\Omega$ e  $C_2  = (102\pm 4 )nF$

\section*{Dichiarazione}
I firmatari di questa relazione dichiarano che il contenuto della relazione \`e originale, con misure effettuate dai membri del gruppo, e che tutti i firmatari hanno contribuito alla elaborazione della relazione stessa.

\end{document}